\documentclass[11pt]{article}
% Basic packages
\usepackage[utf8]{inputenc}
\usepackage[italian]{babel}
\usepackage{hyperref}
% Mathematical packages
\usepackage{amsmath,amssymb,amsthm}
% Layout and formatting packages
\usepackage{geometry}
\usepackage{multicol}
\usepackage{fancyhdr}
\usepackage{titlesec}
% Color and box packages
\usepackage{xcolor}
\usepackage{tcolorbox}
\tcbuselibrary{theorems,skins,breakable}

% Page settings
\geometry{margin=2.5cm}

% Header settings
\pagestyle{fancy}
\fancyhf{}
\fancyhead[L]{Crittografia}
\fancyhead[R]{\thepage}

% Title customization
\titleformat{\section}{\Large\bfseries}{\thesection}{1em}{}
\titleformat{\subsection}{\large\bfseries}{\thesubsection}{1em}{}

% Colored environment definitions
\newtcbtheorem{theorem}{Teorema}%
{colback=blue!5,colframe=blue!70!black,fonttitle=\bfseries,rounded corners}{thm}

\newtcbtheorem{definition}{Definizione}%
{colback=green!5,colframe=green!70!black,fonttitle=\bfseries,rounded corners}{def}

\newtcbtheorem{lemma}{Lemma}%
{colback=orange!5,colframe=orange!70!black,fonttitle=\bfseries,rounded corners}{lem}

\newtcbtheorem{example}{Esempio}%
{colback=purple!5,colframe=purple!70!black,fonttitle=\bfseries,rounded corners}{ex}

\newtcbtheorem{question}{Domanda}%
{colback=red!5,colframe=red!70!black,fonttitle=\bfseries,rounded corners}{q}

\newtcolorbox{problem}[2][]{
enhanced,
colback=cyan!5,
colframe=cyan!70!black,
fonttitle=\bfseries,
rounded corners,
title={Problema: #2},
segmentation style={dashed,cyan!70!black,line width=1pt},
#1
}

% Custom commands
\newcommand{\calU}{\mathcal{U}}

% Spacing settings
\setlength{\parindent}{0pt}      % No indentation
\setlength{\parskip}{1em}        % Space between paragraphs

% Usage examples:
% 
% \begin{theorem}{Nome}{label}
% Testo del teorema
% \end{theorem}
% 
% \begin{definition}{Nome}{label}
% Testo della definizione
% \end{definition}
% 
% \begin{lemma}{Nome}{label}
% Testo del lemma
% \end{lemma}
%
% \begin{example}{Nome}{label}
% Testo dell'esempio
% \end{example}
%
% \begin{question}{Nome}{label}
% Testo della domanda
% \end{question}
%
% \begin{problem}{Nome del problema}
% Testo del problema da risolvere
% \tcblower
% \textbf{Soluzione:} Spiegazione della soluzione
% \end{problem}
%

\begin{document}

% Titolo
\begin{center}
    \LARGE \textbf{Crittografia}
\end{center}

\vspace{0.5cm}
\hrule
\vspace{0.5cm}

\tableofcontents

\vspace{0.5cm}
\hrule
\vspace{0.5cm}

\section{Lezione 1 - 26/09/2023}

\subsection{Una suddivisione approssimativa della Matematica}

    \begin{multicols}{2}
    \textbf{Matematica delle grandezze continue}
    \begin{itemize}
        \item Analisi
        \item Geometria  
        \item Meccanica
    \end{itemize}
    
    \columnbreak
    
    \textbf{Matematica delle grandezze discrete}
    \begin{itemize}
        \item \textcolor{red}{Aritmetica}
        \item \textcolor{red}{Algebra}
        \item Combinatoria
    \end{itemize}
    \end{multicols}
    
\subsection{Gruppi ciclici finiti: l'equazione $z^n = 1$}
    Sia $n$ un intero positivo fissato. Consideriamo l'insieme $\calU_n$ delle soluzioni complesse dell'equazione $z^n = 1$. È facile vedere che le radici sono distinte.
    
    \begin{theorem}{}{}
        Un polinomio ha radici distinte sse è primo con la sua derivata
    \end{theorem}
    
    $f(z) = z^n-1 \implies f'(z) = n z^{n-1}$ e quindi i polinomi $f$ ed $f'$ non hanno fattori comuni.
    
    Per il Teorema fondamentale dell'algebra, $|\calU_n| = n$
    
    Gli elementi di $\calU_n$ sono disposti ai vertici del poligono regolare con $n$ lati, il centro nell'origine ed un vertice in $z = 1$.
    
    Poniamo
    $$
    \delta_n = e^{2\pi i /n} = \cos \left(\frac{2\pi}{n}\right) + i \, \sin\left(\frac{2\pi}{n}\right) = \cos(\theta_n) + i \, \sin(\theta_n)
    $$
    dove
    $$
    i^2 = -1 \quad \text{e} \quad \theta_n = \frac{2\pi}{n}
    $$
    
    Osserviamo che:
    $$
    \delta_n^n=e^{2\pi i} = 1
    $$
    
    Le $n$ soluzioni dell'equazione sono
    $$
    z_j = \delta_n^j = e^{2 \pi i j / n} \qquad \text{per } j=0,\dots, n-1
    $$

    $$
    \delta_n^j \cdot \delta_n^k =
    \begin{cases}
        \delta_n^{j+k} & \text{se } j+k < n \\
        \delta_n^{j+k-n} & \text{se } j+k \geq n
    \end{cases}
    $$

    $$
    \delta_n^{-j} = (\overline{\delta_n})^j =
    \begin{cases}
        \delta_n^{n-j} & \text{se } j \neq 0\\
        \delta_n^0 = 1 & \text{se } j = 0
    \end{cases}
    $$

    Se $m = q \cdot n + r$ dove $0 \leq r < n$ allora

    $$
        \delta_n^m = \delta_n^{qn+r} = (\delta_n^n)^q \cdot \delta_n^r = 1^q \cdot \delta_n^r = \delta_n^r
    $$

    Le soluzioni si possono classificare a seconda di quale sia il più piccolo poligono regolare su cui giacciono, cioè il loro periodo (la potenza da applicare per ottenere 1).

    Es. $\delta_{12}^8$ è di ordine 3. Infatti $(\delta_{12}^8)^3 = \delta_{12}^{24} = \delta_{12}^0 = 1$

    Alcune soluzioni di $z^n = 1$ soddisfano anche $z^k = 1$ per qualche $k$ intero positivo con $k<n$

\subsection{Ordine di un elemento di $\calU_n$}
    L'ordine di $z \in \calU_n$ è il minimo intero positivo $m$ per cui $z_m = 1$

    E' il minimo periodo della successione
    $$1, z, z^2, z^3, \dots, z^{n-1}, z^n = 1, z^{n+1} = z, \dots$$

    $m$ non può superare $n$.

    \begin{definition}{}{}
        Diciamo che un intero $m$ divide un intero $n$ se esiste un intero $a$ t.c. $n = m \cdot a$.
        In questo caso scriviamo $m | n$
    \end{definition}

    \begin{example}{}{}
        Con questa definizione si ha:
        $$
        \begin{cases}
            1 | n & \forall n \in \mathbb{Z} \\
            n | 0 & \forall n \in \mathbb{Z} \\
            0 | n & \iff n = 0 \\
        \end{cases}
        $$
    \end{example}
    

    L'ordine $m$ di $z \in \calU_n$ è un divisore di n: se $z^n=1$ e $z^m=1$, allora $n = q \cdot m + r$ con $0 \leq r < m$ da cui

    $$z^n = z^{qm+r} = (z^m)^q \cdot z^r = z^r = 1$$
    che è assurdo se $r>0$. L'unica possibilità è che $r=0$.

    Se l'ordine di $z\in \calU_n$ è $n$, si dice che $z$ è un generatore poiché le sue potenze successive sono distinte e forniscono tutti gli elementi di $\calU_n$.

    Infatti, se esistessero $j$ e $k$ con $0 \leq j < k < n$ t.c.
    $$z^j = z^k \implies z^{k-j}=1 \quad \text{ma } 1 \leq k-j < n$$
    che è assurdo.

    Dunque gli $n$ elementi indicati sono tutti e soli gli elementi di $calU_n$.

    Se $z$ non è un generatore, questo non accade.

        \begin{definition}{Sottogruppo di un elemento di $\calU_n$}{}
            Dato $z \in \calU_n$ indicheremo con $\langle z \rangle$ il sottoinsieme $\{1, z, z^2, z^3, \dots \} \subseteq \calU_n$ e lo chiameremo \textcolor{red}{sottogruppo generato da z}.
        \end{definition}

        \begin{theorem}{Teorema (Lagrange)}{}
            La cardinalità di $\langle z \rangle$ è uguale all'ordine di z ed è un divisore della cardinalità di $\calU_n$ che è $n$
        \end{theorem}

    Gli insiemi $\calU_n$ con $n$ primo sono speciali!

    Infatti, in questo caso l'ordine di un elemento $z \in \calU_n$ può essere solamente $1$ o $n$ stesso

    Ma solo $z = 1$ ha ordine 1, dunque tutti gli altri elementi di $\calU_n$ hanno ordine $n$ e sono perciò generatori di $\calU_n$

    \begin{example}{}{}
        Dato che $z_7^{12} = 1$, le potenze successive di $z_7$ sono invece
        $$\begin{array}{llll}
            z_7^0 = z_0 & z_7^1 = z_7 & z_7^2 = z_2 & z_7^3 = z_9 \\
            z_7^4 = z_4 & z_7^5 = z_{11} & z_7^6 = z_6 & z_7^7 = z_1 \\
            z_7^8 = z_8 & z_7^9 = z_3 & z_7^{10} = z_{10} & z_7^{11} = z_5
        \end{array}$$
        cioè abbiamo trovato un modo per rimescolare gli elementi di $\calU_n$
    \end{example}

    Osserviamo che tutti gli elementi di $\calU \backslash \{1\}$ sono generatori e dunque la successione $(\delta_{11}^a)^n$ ha periodo 11 per ogni $a \in \{1, 2, \dots, 10\}$. Cioè, i resti di $a, 2a, 3a, \dots$ divisi per $11$ hanno periodo $11$
    
\subsection{L'algoritmo di Euclide}
    \begin{definition}{}{}
        Dati due interi $n$ ed $m$ non entrambi nulli, indichiamo con $(n, m)$ il loro massimo comune divisore.
        $$(n,m) = max\{d \in \mathbb{N}: d | n \quad \land \quad d|m\}$$
    \end{definition}

    Se $z \in \calU_n \cap \calU_m$ (cioè se $z^n = z^m = 1$) allora
    $$z^{\lambda n+\mu m} = (z^n)^\lambda \cdot (z^m)^\mu = 1 \qquad \forall \lambda, \mu \in \mathbb{Z}$$

    Quindi $z^{(n,m)}=1$

    Questo fatto segue dall'algoritmo di Euclide: dati due interi $n$ e $m$, dimostreremo che l'insieme $\{\lambda n + \mu m : \lambda, \mu \in \mathbb{Z}\}$ coincide con l'insieme dei multipli di $d = (n,m)$

    \begin{example}{}{}
        Se $z^{51} = z^{120} = 1$ allora
        $$\begin{array}{rcl}
            120 = 2 \cdot 51 + 18 & \implies & 1 = z^{120} = (z^{51})^2 \cdot z^{18} = z^{18} \\
            51 = 2 \cdot 18 + 15 & \implies & 1 = z^{51} = (z^{18})^2 \cdot z^{15} = z^{15} \\
            18 = 1 \cdot 15 + 3 & \implies & 1 = z^{18} = (z^{15})^1 \cdot z^{3} = z^{3} \\
            15 = 5 \cdot 3 + 0
        \end{array}$$

        Come ricavare i coefficienti $3$ e $-7$ per cui $(51, 120) = 3 = 3 \cdot 120 -7 \cdot 51$?

        $$\begin{array}{rcl}
            18 = 1 \cdot 15 + 3 & \implies & 3 = 18 - 15 \\
            51 = 2 \cdot 18 + 15 & \implies & 3 = 18 - (51 - 2 \cdot 18) \\
            120 = 2 \cdot 51 + 18 & \implies & 3 = (120 - 2 \cdot 51) - (51 - 2 \cdot (120 - 2 \cdot 51)) \\
            & & = 120 - 2 \cdot 51 -51 + 2 \cdot 120 - 4 \cdot 51 \\
            & & = 3 \cdot 120 - 7 \cdot 51
        \end{array}$$
    \end{example}

    \href{https://github.com/0-Sam-0/Cryptography-UniPR}{Mio codice Python che implementa l'algoritmo di Euclide}.

    \begin{question}
    {Quali sono i generatori di $\calU_n$?}{gen-def}
        I generatori di $\calU_n$ sono gli elementi di ordine $n$, cioè quegli elementi $z \in \calU_n$ tali che le loro potenze successive $z^0, z^1, z^2, \ldots, z^{n-1}$ sono tutte distinte e coincidono con tutti gli elementi di $\calU_n$.
    \end{question}

    \begin{question}
    {Quanti sono i generatori di $\calU_n$?}{gen-count}
        Il numero di generatori di $\calU_n$ è uguale al numero di interi $k$ con $1 \leq k \leq n-1$ tali che $(k,n) = 1$, cioè $\phi(n)$ dove $\phi$ è la funzione di Eulero.
    \end{question}

    \begin{question}
    {Dato un generatore di $\calU_n$, come si trovano tutti gli altri?}{gen-find}
        Se $g$ è un generatore di $\calU_n$, allora tutti i generatori sono esattamente gli elementi della forma $g^k$ dove $1 \leq k \leq n-1$ e $(k,n) = 1$.
    \end{question}

\subsection{Gruppi}
    \begin{definition}{}{}
        L'insieme $G$ si dice gruppo rispetto all'operazione $\circ$ se
        \begin{itemize}
            \item $\forall g, h \in G: g \circ h \in G$
            \item $\exists e \in G : \forall g \in G :g \circ e = e \circ g = g$
            \item $\forall g \in G: \exists h \in G : g \circ h = h \circ g = e$
            \item $\forall g, h, j \in G : (g \circ h) \circ j = g \circ (h \circ j)$
        \end{itemize}

        L'elemento $e$ si dice elemento neutro o unità di $G$.

        L'elemento $h$ si dice inverso o reciproco di $g$ e si indica con $g^{-1}$.

        Un sottoinsieme $H \subseteq G$ che sia a sua volta un gruppo rispetto all'operazione $\circ$ si dice sottogruppo di $G$.
        
    \end{definition}

    \begin{definition}{}{}
        Sia $G$ un gruppo rispetto all'operazione $\circ$
        \begin{itemize}
            \item se $g \circ h = h \circ g \forall g,h \in G$ il gruppo si dice abeliano o commutativo
            \item se $\exists g \in G : G = \langle g \rangle$ allora $G$ si dice ciclico
        \end{itemize}
        In questo caso definiamo $\langle g \rangle = \{ g^n : n \in \mathbb{Z}\}$
            
        La definizione alternativa è necessaria quando $G$ è infinito
        
    \end{definition}

    L'insieme $\calU_n$ è un gruppo abeliano ciclico rispetto alla moltiplicazione, ed è generato da $\delta_n$

    L'insieme $\calU_m$ ne è un sottogruppo sse $m|n$, ed è generato da $\delta_n^{n/m}=\delta_m$

    \begin{example}{}{}
        L'insieme $\mathbb{Z}$ è un gruppo ciclico infinito generato da $g \in \{\pm 1\}$
    \end{example}
    
    % Perchè non si possono creare gruppi ciclici non abeliani?
    
    \begin{problem}{}{}
        Se $G = \langle g \rangle$ e $card(G) = n$, quante soluzioni ha $x^d = 1$? (\textit{$d$ fissato})
        \tcblower
        Se $x$ è una soluzione, sia $x = g^m$ per un intero $m \in \{0, \dots, n-1\}$

        L'equazione diventa $g^{md} = 1$ e da questo segue che $n|md$

        Dividendo per $(n,d)$ si ha $\frac{n}{(n,d)}|m$ e il numero di $m \in \{0, \dots, n-1\}$ che la soddisfano è $(n,d)$.
        \begin{example}{}{}
            Risolvere $x^4 = 1$ in $\calU_{12}$.

            $\begin{array}{lllll}
                \begin{cases}
                    m \in \{0,1,\dots,11\} \\
                    g^{4m} = 1 \\
                    g^{12} = 1
                \end{cases} &
                \implies &
                12|4m \implies 3|m &
                \implies &
                m \in \{0,3,6,9\}
            \end{array}$

            \tcblower
            
            Risolvere $x^4 = 1$ in $\calU_{6}$.
            
            $\begin{array}{lllll}
                \begin{cases}
                    m \in \{0,1,\dots,5\} \\
                    g^{4m} = 1 \\
                    g^{6} = 1
                \end{cases} &
                \implies &
                6|4m \implies 3|2m &
                \implies &
                m \in \{0,3\}
            \end{array}$
        \end{example}
    \end{problem}

\subsection{Isomorfismi fra gruppi}
    \begin{definition}{}{}
        Siano $G$ e $G'$ due gruppi, con operazione $\circ$ e $*$ rispettivamente.
        
        Un'applicazione biiettiva $\phi : G \rightarrow G'$ si dice isomorfismo se
        $$\phi(x \circ y) = \phi(x) * \phi(y) \quad \forall x,y \in G$$
    \end{definition}

    \begin{example}{}{}
        Posto $G = \mathbb{R}^+$ con l'operazione di moltiplicazione e $G'=\mathbb{R}$ con l'operazione di addizione, un isomorfismo è la funzione $\phi(x) = log(x)$. Infatti: $log(x \cdot y) = log(x) + log(y)$
    \end{example}
    
    Un isomorfismo non è un'uguaglianza e ciò ha delle conseguenze computazionali.
    Certi insiemi non sono adatti alla crittografia, ma degli insiemi a loro isomorfi sì.

    Gli isomorfismi sono invertibili, ma ciò non significa che le operazioni coinvolte abbiano la stessa complessità computazionale.

    \begin{example}{}{}
        Dato $\phi : \mathbb{R}^+ \rightarrow \mathbb{R}$ dell'esempio precedente, l'inverso $\psi: \mathbb{R} \rightarrow \mathbb{R}^+$ è l'esponenziale.

        $\phi \circ \psi$ è l'identità su $\mathbb{R}$ e $\psi \circ \phi$ è l'identità su $\mathbb{R}^+$
    \end{example}
    
\vspace{0.5cm}
\hrule
\vspace{0.5cm}

\section{Lezione 2 - 27/09/2023}

\end{document}